\specchapter{LFI-Arm64 Assembly}

% Don't forget to undefine at the bottom
% If you change these, update the xN/wN terminology list.
\providecommand{\xbase}{\code{x21}\xspace}
\providecommand{\xaddr}{\code{x18}\xspace}
\providecommand{\xret}{\code{x30}\xspace}
\providecommand{\xsp}{\code{sp}\xspace}

\informative{LFI-Arm64 Assembly is a pseudo-ISA specified by this document. LFI-Arm64 has significant overlap with the Arm64 ISA.}

\specsection{Terminology}

\specitem
Each \term{instruction} in LFI-Arm64 is specified as a translation to Arm64
instructions. The implementation may choose any translation as long as it
preserves the same memory accesses and final register state as the translations
listed in this document.

\informative{The translations are crafted so that the resulting program is
accepted by the LFI-Arm64 Verifier. The translations listed in this document
are the recommended implementations.}

\specitem
The notation \code{xN} refers to any of the following registers:
\code{x0},
\code{x1},
\code{x2},
\code{x3},
\code{x4},
\code{x5},
\code{x6},
\code{x7},
\code{x8},
\code{x9},
\code{x10},
\code{x11},
\code{x12},
\code{x13},
\code{x14},
\code{x15},
\code{x16},
\code{x17},
\code{x18},
\code{x19},
\code{x20},
\code{x21},
\code{x22},
\code{x23},
\code{x24},
\code{x25},
\code{x26},
\code{x27},
\code{x28},
\code{x29},
\code{x30}.

\specitem
The notation \code{wN} refers to any of the following registers, corresponding to the lower 32 bits of \code{xN}:
\code{w0},
\code{w1},
\code{w2},
\code{w3},
\code{w4},
\code{w5},
\code{w6},
\code{w7},
\code{w8},
\code{w9},
\code{w10},
\code{w11},
\code{w12},
\code{w13},
\code{w14},
\code{w15},
\code{w16},
\code{w17},
\code{w18},
\code{w19},
\code{w20},
\code{w21},
\code{w22},
\code{w23},
\code{w24},
\code{w25},
\code{w26},
\code{w27},
\code{w28},
\code{w29},
\code{w30}.

\specsection{Instructions}

\specitem
If an Arm64 instruction that is permitted by the LFI-Arm64 Verifier is not specified in this document, then its translation is the identity translation (no modification).

\specitem
Instructions that attempt to use \xbase or \xaddr are invalid in LFI-Arm64.

\specitem
Each instruction is defined using a table, with the LFI-Arm64 instruction of the left, and its translation to Arm64 on the right. Some instruction translations may use temporary locations, expressed using register notation---these may be translated to registers or valid memory accesses. 64-bit temporary locations are notated as \code{xT} and \code{xT2}; 32-bit temporary locations are notated as \code{wT} and \code{wT1}.

\informative{An implementation may choose to implement temporary locations via a reserved register, via a register allocation scheme, via stack slots, or possibly some other mechanism or combinations of these mechanisms.}

\specsection{Indirect Branches}

\specitem
\begin{tabular}{@{}p{7cm}p{7cm}@{}}\hline
    LFI-Arm64 & Arm64 \\ \hline
    \code{br xN}        & \makecell{\code{add x18, x21, wN, uxtw} \\ \code{br x18}} \\ \hline
\end{tabular}

\specitem
\begin{tabular}{@{}p{7cm}p{7cm}@{}}\hline
    \code{blr xN}        & \makecell{\code{add x18, x21, wN, uxtw} \\ \code{blr x18}} \\ \hline
\end{tabular}

\specitem
\begin{tabular}{@{}p{7cm}p{7cm}@{}}\hline
    \code{ret}        & \makecell{\code{ret}} \\ \hline
\end{tabular}

\specitem
\begin{tabular}{@{}p{7cm}p{7cm}@{}}\hline
    \code{ret xN}        & \makecell{\code{add x18, x21, wN, uxtw} \\ \code{ret x18}} \\ \hline
\end{tabular}

\specsection{Register Accesses}

\specitem
\begin{tabular}{@{}p{7cm}p{7cm}@{}}\hline
    \code{ldr x30, [...]}        & \makecell{\code{ldr xT, [...]} \\ \code{add x30, x21, wT, uxtw}} \\ \hline
\end{tabular}

\specitem
\begin{tabular}{@{}p{7cm}p{7cm}@{}}\hline
    \code{mov sp, xN}        & \makecell{\code{add sp, x21, xN, uxtw}} \\ \hline
\end{tabular}

\specitem
\begin{tabular}{@{}p{7cm}p{7cm}@{}}\hline
    \code{add sp, sp, \{\#i,xM\}}        & \makecell{\code{add xT, sp, \{\#i,xM\}} \\ \code{add sp, x21, wT, uxtw}} \\ \hline
\end{tabular}

\specitem
\begin{tabular}{@{}p{7cm}p{7cm}@{}}\hline
    \code{sub sp, sp, \{\#i,xM\}}        & \makecell{\code{sub xT, sp, \{\#i,xM\}} \\ \code{add sp, x21, wT, uxtw}} \\ \hline
\end{tabular}

\specsection{Memory Accesses}

\specitem
\begin{tabular}{@{}p{7cm}p{7cm}@{}}\hline
    \code{\{ld,st\}r \ldots, [xM]} & \makecell{\code{\{ld,st\}r \ldots, [x21, wM, uxtw]}} \\ \hline
\end{tabular}

\specitem
\begin{tabular}{@{}p{7cm}p{7cm}@{}}\hline
    \code{\{ld,st\}r \ldots, [xM, \#i]} & \makecell{\code{add xT, xM, \#i} \\ \code{\{ld,st\}r \ldots, [x21, wT, uxtw]}} \\ \hline
\end{tabular}

\specitem
\begin{tabular}{@{}p{7cm}p{7cm}@{}}\hline
    \code{\{ld,st\}r \ldots, [xM, \#i]!} & \makecell{\code{add xM, xM, \#i} \\ \code{\{ld,st\}r \ldots, [x21, wM, uxtw]}} \\ \hline
\end{tabular}

\specitem
\begin{tabular}{@{}p{7cm}p{7cm}@{}}\hline
    \code{\{ld,st\}r \ldots, [xM], \#i} & \makecell{\code{\{ld,st\}r \ldots, [x21, wM, uxtw]} \\ \code{add xM, xM, \#i}} \\ \hline
\end{tabular}

\specitem
\begin{tabular}{@{}p{7cm}p{7cm}@{}}\hline
    \code{\{ld,st\}r \ldots, [xM1, xM2]} & \makecell{\code{add xT, xM1, xM2} \\ \code{\{ld,st\}r \ldots, [x21, wT, uxtw]}} \\ \hline
\end{tabular}

\specitem
\begin{tabular}{@{}p{7cm}p{7cm}@{}}\hline
    \code{\{ld,st\}r \ldots, [xM1, xM2, mod \#i]} & \makecell{\code{add xT, xM1, xM2, mod \#i} \\ \code{\{ld,st\}r \ldots, [x21, wT, uxtw]}} \\ \hline
\end{tabular}

\specitem
\begin{tabular}{@{}p{7cm}p{7cm}@{}}\hline
    \code{\{ld,st\}X \ldots, [xM, \#i]} & \makecell{\code{add x18, x21, wM, uxtw} \\ \code{\{ld,st\}X \ldots, [x18, \#i]}} \\ \hline
\end{tabular}

\specitem
\begin{tabular}{@{}p{7cm}p{7cm}@{}}\hline
    \code{\{ld,st\}X \ldots, [xM, \#i]!} & \makecell{\code{add x18, x21, wM, uxtw} \\ \code{\{ld,st\}X \ldots, [x18]} \\ \code{add xM, xM, \#i}} \\ \hline
\end{tabular}

\specitem
\begin{tabular}{@{}p{7cm}p{7cm}@{}}\hline
    \code{\{ld,st\}X \ldots, [xM], \#i} & \makecell{\code{add x18, x21, wM, uxtw} \\ \code{\{ld,st\}X \ldots, [x18, \#i]} \\ \code{add xM, xM, \#i}} \\ \hline
\end{tabular}

\specitem
\begin{tabular}{@{}p{7cm}p{7cm}@{}}\hline
    \code{\{ld,st\}X \ldots, [xM1], xM2} & \makecell{\code{add x18, x21, wM1, uxtw} \\ \code{\{ld,st\}X \ldots, [x18, \#i]} \\ \code{add xM1, xM1, xM2}} \\ \hline
\end{tabular}

\specsection{Thread-local Storage Accesses}

\specitem
 \begin{tabular}{@{}p{7cm}p{7cm}@{}} \hline
    \code{mrs x0, tpidr\_el0} & \makecell{\code{mov wT, w30} \\ \code{ldr x30, [x21, \#8]} \\ \code{blr x30} \\ \code{add x30, x21, wT, uxtw}} \\ \hline
\end{tabular} 

\specitem
 \begin{tabular}{@{}p{7cm}p{7cm}@{}} \hline
    \code{mrs xN, tpidr\_el0} & \makecell{\code{mov wT1, w30} \\ \code{mov xT2, x0} \\ \code{ldr x30, [x21, \#8]} \\ \code{blr x30} \\ \code{mov xN, x0} \\  \code{mov x0, xT2} \\ \code{add x30, x21, wT1, uxtw}} \\ \hline
\end{tabular} 

\informative{Attempts to read from \code{tpidr\_el0} are translated into calls to rtcall 1}.

\specitem
 \begin{tabular}{@{}p{7cm}p{7cm}@{}} \hline
    \code{mrs tpidr\_el0, x0} & \makecell{\code{mov wT1, w30} \\ \code{ldr x30, [x21, \#16]} \\ \code{blr x30} \\ \code{add x30, x21, wT1, uxtw}} \\ \hline
\end{tabular} 

\specitem
 \begin{tabular}{@{}p{7cm}p{7cm}@{}} \hline
    \code{msr tpidr\_el0, xN} & \makecell{\code{mov wT1, w30} \\ \code{mov xT2, x0} \\ \code{mov x0, xN} \\ \code{ldr x30, [x21, \#16]} \\ \code{blr x30} \\  \code{mov x0, xT2} \\ \code{add x30, x21, wT1, uxtw}} \\ \hline
\end{tabular} 

% TODO: it might make sense to reserve rtcall 0 so that it could be used for direct entry (blr x21).
\informative{Attempts to write \code{tpidr\_el0} are translated into calls to rtcall 2}.

\specsection{System Instructions}

\specitem
 \begin{tabular}{@{}p{7cm}p{7cm}@{}} \hline
    \code{svc \#0} & \makecell{\code{mov wT, w30} \\ \code{ldr x30, [x21]} \\ \code{blr x30} \\ \code{add x30, x21, wT, uxtw}} \\ \hline
\end{tabular} 

\informative{Attempts to execute a system call are translated into calls to rtcall 0}.

\let\xbase\undefined
\let\xaddr\undefined
\let\xret\undefined
\let\xsp\undefined
