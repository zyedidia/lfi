\specchapter{LFI-Arm64 Verifier}

% Don't forget to undefine at the bottom
\providecommand{\xbase}{\code{x21}\xspace}
\providecommand{\xaddr}{\code{x18}\xspace}
\providecommand{\xret}{\code{x30}\xspace}
\providecommand{\xsp}{\code{sp}\xspace}

\informative{The LFI-Arm64 verifier is specified in terms of Arm64 machine code, expressed using GNU assembly syntax.}

\specsection{Terminology}

\specitem
The notation \code{xN} refers to any of the following registers:
\code{x0},
\code{x1},
\code{x2},
\code{x3},
\code{x4},
\code{x5},
\code{x6},
\code{x7},
\code{x8},
\code{x9},
\code{x10},
\code{x11},
\code{x12},
\code{x13},
\code{x14},
\code{x15},
\code{x16},
\code{x17},
\code{x18},
\code{x19},
\code{x20},
\code{x21},
\code{x22},
\code{x23},
\code{x24},
\code{x25},
\code{x26},
\code{x27},
\code{x28},
\code{x29},
\code{x30}.

\specitem
The notation \code{wN} refers to any of the following registers, corresponding to the lower 32 bits of \code{xN}:
\code{w0},
\code{w1},
\code{w2},
\code{w3},
\code{w4},
\code{w5},
\code{w6},
\code{w7},
\code{w8},
\code{w9},
\code{w10},
\code{w11},
\code{w12},
\code{w13},
\code{w14},
\code{w15},
\code{w16},
\code{w17},
\code{w18},
\code{w19},
\code{w20},
\code{w21},
\code{w22},
\code{w23},
\code{w24},
\code{w25},
\code{w26},
\code{w27},
\code{w28},
\code{w29},
\code{w30}.

\specitem
An instruction \textit{writes} to register \code{xN} if it modifies the
64-bit value stored in \code{xN}, including by modification via the name
\code{wN}.

\specsection{Register Accesses}

\specitem
No instruction may write to register \xbase.

\specitem
Unless otherwise specified\footnote{Pre/post-increment memory operands (\ref{specitem:a64_mem}).}, no instruction except the following may write to register \xaddr.

\begin{itemize}
    \item \code{add \xaddr, \xbase, wN, uxtw}
\end{itemize}

\specitem
No instruction except the following may write to register \xret.

\begin{itemize}
    \item \code{add \xret, \xbase, wN, uxtw}

    \item \code{ldr \xret, [\xbase, \#i]}, where \code{i \% 8 == 0} and \code{i < 256}.
\end{itemize}

\specitem
Unless otherwise specified\footnote{Pre/post-increment memory operands (\ref{specitem:a64_mem}).}, no instruction except the following may write to register \xsp.

\begin{itemize}
    \item \code{add \xsp, \xbase, wN, uxtw}
\end{itemize}

\specsection{Control Flow}

\specitem
The \code{br} instruction must target \xaddr.

\specitem
The \code{blr} instruction must target \xaddr or \xret.

\specitem
The \code{ret} instruction must target \xret.

\specsection{Memory Accesses}

\specitem
\label{specitem:a64_mem}
A memory operand must be one of the following forms:

\begin{enumerate}
    \item \code{[\xaddr, \#i]}
    \item \code{[\xsp, \#i]}
    \item \code{[\xsp, \#i]!}
    \item \code{[\xsp], \#i}
    \item \code{[\xbase, wN, uxtw]}
    \item \code{[\xbase, \#i]}, where \code{i \% 8 == 0} and \code{i < 256}.
\end{enumerate}

\specsection{System Register Accesses}

\specitem
Use of the \code{msr} and \code{mrs} instructions is not permitted.

\specsection{Permitted Instructions}

\specitem
All Armv8.1 system instructions are not permitted (\code{svc}, \code{hvc}, \code{clrex}, \code{hlt}, \code{eret}, \ldots). All remaining Armv8.1 instructions are permitted.

\let\xbase\undefined
\let\xaddr\undefined
\let\xret\undefined
\let\xsp\undefined
